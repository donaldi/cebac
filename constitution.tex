\documentclass[a4paper,11pt,onecolumn ]{article}

% NB this file is under git version control at 
% https://github.com/donaldi/cebac
% so if you have not downloaded it from there you may not have
% the latest version!

%extend lists
\usepackage{enumitem}

% lets sections/subsections be un-numbered and yet still in toc
% note that I've slightly modified this to work for subsecs too
\usepackage[silent]{unnumberedtotoc}

% use images
\usepackage{graphicx}
\graphicspath{ {./} }

%hyperlink refs
\usepackage{hyperref}
\usepackage{xcolor}
\definecolor{ultramarine}{RGB}{0,32,96} 
\hypersetup{
    colorlinks = true,
    urlcolor = ultramarine,
    linkcolor = ultramarine,
    citecolor = red
}

%\usepackage{draftwatermark}
%\SetWatermarkText{Draft}
%\SetWatermarkScale{5}

\title{CONSTITUTION OF COMANN EACHDRAIDH SGÌRE A’ BHAC SCIO}
\pagestyle{plain}

\newlist{legal}{enumerate}{10}

% set "legal" list (cloned from enumerate) to use arabic bullets, have no left-margin and 
% in references (ref)to use the number without the period after it.
\setlist[legal]{label*=\arabic*.,ref=\arabic*,left=0pt}
% make the main global list resumable
\setlist[legal,1]{resume,widest=100}


\begin{document}
\maketitle

\begin{center}
\includegraphics[width=0.3\paperwidth]{cebac-logo}
\end{center}

\pagebreak

\tableofcontents

\pagebreak

\addsec{GENERAL}

\addsubsec{Type of organisation}

\begin{legal}
\item The organisation will, upon registration, be a Scottish Charitable Incorporated Organisation (SCIO).
\end{legal}

\addsubsec{Scottish principal office}

\begin{legal}
\item The principal office of the organisation will be in Scotland (and must remain in Scotland).
\end{legal}

\addsubsec{Name}

\begin{legal}
\item The name of the organisation is \newline COMANN EACHDRAIDH SGÌRE A’ BHAC SCIO.
\end{legal}


\addsubsec{Purposes}

\begin{legal}
\item The organisation’s purposes are to benefit the community in the District of Back in the Isle of Lewis, in Scotland, as defined by the following postcodes: 

\begin{quote}
HS2 0JP, HS2 0JR, HS2 0JS, HS2 0JN, HS2 0JT, HS2 0JY, HS2 0JZ, HS2 0LA, HS2 0LB, HS2 0LD, HS2 0LE, HS2 0LF, HS2 0LH, HS2 0LJ, HS2 0LQ, HS2 0LN, HS2 0LP, HS2 0LR, HS2 0LS, HS2 0LT, HS2 0LW, HS2 0NA, HS2 0NB
\end{quote}

and any members from outwith this area with an interest in the history and culture of the area, in the furtherance of the following:
\begin{legal}
\item The advancement of education through educational events and publications, building the community knowledge base.
\item The advancement of health through social activity, promoting good physical and mental health within the community.
\item The advancement of citizenship and community development by participation benefiting the wider society within the district.
\item The advancement of arts, heritage, culture by maintaining knowledge of the aforementioned community promoting community identity.  
\item The promotion of equality and diversity, inviting the whole community to learn about the history.
\end{legal}

\end{legal}

\addsubsec{Powers}

\begin{legal}
\item The organisation has power to do anything which is calculated to further its purposes or is conducive or incidental to doing so.
\item The income and property of the organisation shall be applied solely towards promoting the Purposes and do not belong to the Members. Any surplus income or assets of the Organisation is/are to be applied for the benefit of the Community.
\end{legal}

\addsubsec{Liability of members}

\begin{legal}
\item \label{clause:liability} The members of the organisation have no liability to pay any sums to help to meet the debts (or other liabilities) of the organisation if it is wound up; accordingly, if the organisation is unable to meet its debts, the members will not be held responsible.
 
\item The members and charity trustees have certain legal duties under the Charities and Trustee Investment (Scotland) Act 2005; and clause \ref{clause:liability} does not exclude (or limit) any personal liabilities they might incur if they are in breach of those duties or in breach of other legal obligations or duties that apply to them personally.
\end{legal}

\addsubsec{General structure}
\begin{legal}
\item The structure of the organisation consists of:
    	\begin{legal}
         \item the MEMBERS - who have the right to attend members' meetings (including any annual general meeting) and have important powers under the constitution; in particular, the members appoint people to serve on the board and take decisions on changes to the constitution itself;
            \begin{legal}
             \item the minimum number of ordinary members shall be 20
            \end{legal}
        \item the BOARD - who hold regular meetings, and generally control the activities of the organisation; for example, the board is responsible for monitoring and controlling the financial position of the organisation.
       \end{legal}

\item The people serving on the board are referred to in this constitution as CHARITY TRUSTEES. 
\end{legal}

\addsec{MEMBERS}
\addsubsec{Qualifications for membership}

\begin{legal} %clause11
\item \label{clause:memqualification} Qualifications for membership: those individuals aged 16 and over who:

    \begin{legal}[label=\alph*)]
    \item are resident in the community of Sgìre a’ Bhac or are entitled to vote at a local government election in the Loch a Tuath Ward; and
    \item support the purposes.
    \end{legal}
    
    \begin{legal}
    \item Associate Members are:
        \begin{legal}[label=\alph*)]
        \item individuals who are not eligible to vote in a local government election in the Loch a Tuath Ward OR 
        \item groups (wherever located)
        \end{legal}
    who support the purposes.

    Associate Members are neither eligible to stand for election to the Board nor to vote at any AGM or GM.

    \item Junior Members. Membership applications will be accepted from members of the community who:
        \begin{legal}[label=\alph*)]
        \item are aged between 5 and 15 and 
        \item support the purposes.
        \end{legal}

    Junior members are neither eligible to become a charity trustee by any means, nor to vote at any AGM or GM. 

    \item	Declaring that, if a member ceases to comply with any of the criteria of clauses \ref{clause:memqualification}, \ref{clause:memqualification}.1 or \ref{clause:memqualification}.2 they will be obliged to inform the organisation and will thereafter have membership reclassified in terms of either clause \ref{clause:memqualification}, \ref{clause:memqualification}.1 or \ref{clause:memqualification}.2 and that if the organisation becomes aware of changes itself it will so reclassify the member and notify them accordingly.

    \item Membership of the organisation may not be transferred by a member.
    \end{legal}
\end{legal}


\addsubsec{Application for membership}

\begin{legal} %clause12
\item Any person who wishes to become a member must submit a written or online application for membership; the application will then be considered by the board at its next board meeting.
\end{legal}


\addsubsec{Membership subscription}

\begin{legal} %clause13
\item No membership subscription will be payable.
\end{legal}

\addsubsec{Register of members}

\begin{legal}
%clause14
\item The board must keep a register of members, setting out 
    \begin{legal}
        \item for each current member:
            \begin{legal}
            \item his/her full name and address; and
            \item the date on which he/she was registered as a member of the organisation; 
            \end{legal}
        \item for each former member - for at least six years from the date on he/she ceased to be a member:
            \begin{legal}
            \item his/her name; and
            \item the date on which he/she ceased to be a member. 
            \end{legal}
    \end{legal}
%clause15
\item The board must ensure that the register of members is updated within 28 days of any change:
    \begin{legal}
        \item which arises from a resolution of the board or a resolution passed by the members of the organisation; or 
        \item which is notified to the organisation.
    \end{legal}

%16
\item If a member or charity trustee of the organisation requests a copy of the register of members, the board must ensure that a copy is supplied to him/her within 28 days, providing the request is reasonable; if the request is made by a member (rather than a charity trustee), the board may provide a copy which has the addresses blanked out.
\end{legal}

\addsubsec{Withdrawal from membership}
\begin{legal}
%17
\item Any person who wants to withdraw from membership must give a written notice of withdrawal to the organisation, signed by him/her; he/she will cease to be a member as from the time when the notice is received by the organisation.
\end{legal}

\addsubsec{Transfer of membership}
\begin{legal}
%18
\item Membership of the organisation may not be transferred by a member.
\end{legal}

\addsubsec{Re-registration of members}
\begin{legal}
%19
\item \label{clause:confirmmem} The board may, at any time, issue notices to the members requiring them to confirm that they wish to remain as members of the organisation, and allowing them a period of 28 days (running from the date of issue of the notice) to provide that confirmation to the board. 
\item \label{clause:failtoconfirmmem} If a member fails to provide confirmation to the board (in writing or by e-mail) that he/she wishes to remain as a member of the organisation before the expiry of the 28-day period referred to in clause \ref{clause:confirmmem}, the board may expel him/her from membership.

%21
\item A notice under clause \ref{clause:confirmmem} will not be valid unless it refers specifically to the consequences (under clause \ref{clause:failtoconfirmmem}) of failing to provide confirmation within the 28-day period.
\end{legal}

\addsubsec{Expulsion from membership}
\begin{legal}
%22
\item \label{clause:expel_members} Any person may be expelled from membership by way of a resolution passed by not less than two thirds of those present and voting at a members' meeting, providing the following procedures have been observed:
    \begin{legal}
    \item at least 21 days’ notice of the intention to propose the resolution must be given to the member concerned, specifying the grounds for the proposed expulsion;
    \item the member concerned will be entitled to be heard on the resolution at the members' meeting at which the resolution is proposed.
    \end{legal}
\end{legal}

\addsubsec{Termination}

\begin{legal}
\item Membership of the organisation will terminate on death.
\end{legal}

\addsec{DECISION-MAKING BY THE MEMBERS}
\addsubsec{Members’ meetings}

\begin{legal}
%24
\item \label{clause:agm} The board must arrange a meeting of members (an annual general meeting or ``AGM'') in each calendar year.
\item The gap between one AGM and the next must not be longer than 15 months.
\item Notwithstanding clause \ref{clause:agm}, an AGM does not need to be held during the calendar year in which the organisation is formed; but the first AGM must still be held within 15 months of the date on which the organisation is formed. 
\item The business of each AGM must include
    \begin{legal}
    \item a report by the chair on the activities of the organisation;
    \item consideration of the annual accounts of the organisation;
    \item the election/re-election of charity trustees, as referred to in clauses \ref{clause:agm_elect_trustee} to \ref{clause:trustee_re-election}.
    \end{legal}
\item The board may arrange a special members' meeting at any time.
\end{legal}

\addsubsec{Power to request the board to arrange a special members’ meeting}

\begin{legal}
%29
\item \label{clause:spec-mem-meeting} The board must arrange a special members’ meeting if they are requested to do so by a notice (which may take the form of two or more documents in the same terms, each signed by one or more members) by members who amount to 5\% or more of the total membership of the organisation at the time, providing:
    \begin{legal}
    \item the notice states the purposes for which the meeting is to be held; and
    \item those purposes are not inconsistent with the terms of this constitution, the Charities and Trustee (Investment) Scotland Act 2005 or any other statutory provision.
    \end{legal}
%30
\item If the board receive a notice under clause \ref{clause:spec-mem-meeting}, the date for the meeting which they arrange in accordance with the notice must not be later than 28 days from the date on which they received the notice.
\end{legal}

\addsubsec{Notice of members’ meetings}
\begin{legal}
%31
\item \label{clause:agm_notice} At least 14 clear days’ notice must be given of any AGM or any special members' meeting.
%32
\item If members are to be permitted to participate in a meeting by way of audio and/or audio-visual link(s), the members must, in advance of the meeting, be provided with details of how to connect and participate via that link or links; and (particularly for the benefit of those members who may have difficulties in using a computer or laptop for this purpose) the members attention should be drawn to the following options:
    \begin{legal}[label=\alph*)]
    \item participating in the meeting via an audio link accessed by phone, using dial-in details (if that forms part of the arrangements);
    \item (where attendance in person is to be permitted, either on an open basis or subject to a restriction on the total number who will be permitted to attend) the ability to attend the meeting in person
    \end{legal}
%33
\item The notice calling a members' meeting must specify in general terms what business is to be dealt with at the meeting; and
    \begin{legal}
    \item in the case of a resolution to alter the constitution, must set out the exact terms of the proposed alteration(s); or
    \item in the case of any other resolution falling within clause \ref{clause:two_thirds_maj} (requirement for two-thirds majority) must set out the exact terms of the resolution.
    \end{legal}
%34
\item The reference to “clear days” in clause \ref{clause:agm_notice} shall be taken to mean that, in calculating the period of notice,
    \begin{legal}
        \item the day after the notices are posted (or sent by e-mail) should be excluded; and 
        \item the day of the meeting itself should also be excluded.
    \end{legal}
%35
\item Notice of every members' meeting must be given to all the members of the organisation, and to all the charity trustees; but the accidental omission to give notice to one or more members will not invalidate the proceedings at the meeting.
%36
\item Any notice which requires to be given to a member under this constitution must be:
    \begin{legal}
        \item sent by post to the member, at the address last notified by him/her to the organisation; or 
        \item sent by e-mail to the member, at the e-mail address last notified by him/her to the organisation.
    \end{legal}
\end{legal}

\addsubsec{Procedure at members’ meetings}
\begin{legal}
%37
\item No valid decisions can be taken at any members' meeting unless a quorum is present. 
%38
\item The quorum for a members' meeting is, whichever is greater: 8 ordinary members, or 10\% of the ordinary members, present in person or via proxy or the internet.
%39 
\item If a quorum is not present within 15 minutes after the time at which a members' meeting was due to start - or if a quorum ceases to be present during a members' meeting - the meeting cannot proceed; and fresh notices of meeting will require to be sent out, to deal with the business (or remaining business) which was intended to be conducted. 
%40
\item The chair of the organisation should act as chairperson of each members' meeting.
%41
\item If the chair of the organisation is not present within 15 minutes after the time at which the meeting was due to start (or is not willing to act as chairperson), the charity trustees present at the meeting must elect (from among themselves) the person who will act as chairperson of that meeting.
\end{legal}

\addsubsec{Voting at members’ meetings}
\begin{legal}
%42
\item Every charity trustee shall have one vote in his/her capacity as a member, which (whether on a show of hands or on a secret ballot) may be given either personally or by proxy.
%43
\item All decisions at members' meetings will be made by majority vote - with the exception of the types of resolution listed in clause \ref{clause:two_thirds_maj}.
%44
\item \label{clause:two_thirds_maj} The following resolutions will be valid only if passed by not less than two thirds of those voting on the resolution at a members’ meeting (or if passed by way of a written resolution under clause 48):
    \begin{legal}
        \item a resolution amending the constitution;
        \item a resolution expelling a person from membership under clause \ref{clause:expel_members};
        \item a resolution directing the board to take any particular step (or directing the board not to take any particular step);
        \item a resolution approving the amalgamation of the organisation with another SCIO (or approving the constitution of the new SCIO to be constituted as the successor pursuant to that amalgamation);
        \item a resolution to the effect that all of the organisation’s property, rights and liabilities should be transferred to another SCIO (or agreeing to the transfer from another SCIO of all of its property, rights and liabilities);
        \item a resolution for the winding up or dissolution of the organisation.
    \end{legal}
%45
\item If there is an equal number of votes for and against any resolution, the chairperson of the meeting will be entitled to a second (casting) vote.
%46
\item A resolution put to the vote at a members' meeting will be decided on a show of hands - unless the chairperson (or at least two other members present at the meeting) ask for a secret ballot.
\item The chairperson will decide how any secret ballot is to be conducted, and he/she will declare the result of the ballot at the meeting.
\end{legal}

\addsubsec{Written resolutions by members}
\begin{legal}
%48
\item A resolution agreed to in writing (or by e-mail) by all the members will be as valid as if it had been passed at a members’ meeting; the date of the resolution will be taken to be the date on which the last member agreed to it.
\end{legal}

\addsubsec{Minutes}
\begin{legal}
%49
\item \label{clause:keep_minutes} The board must ensure that proper minutes are kept in relation to all members' meetings.
%50
\item Minutes of members' meetings must include the names of those present; and (so far as possible) should be signed by the chairperson of the meeting.
%51
\item The board shall make available copies of the minutes referred to in clause \ref{clause:keep_minutes} to any member of the public requesting them; but on the basis that the board may exclude confidential material to the extent permitted under clause \ref{clause:minutes_confidential}.
\end{legal}

\addsec{BOARD}
\addsubsec{Number of charity trustees}

\begin{legal}
%52
\item \label{clause:max_trustees} The maximum number of charity trustees is 15; out of that:
    \begin{legal}
        \item no more than 12 shall be charity trustees who were elected\slash appointed under clauses \ref{clause:agm_elect_trustee} and \ref{clause:board_appoint_trustee} (or deemed to have been appointed under clause \ref{clause:initial_trustees}); and 
        \item no more than 3 shall be charity trustees who were co-opted under the provisions of clauses \ref{clause:board_co-opt_trustee} and \ref{clause:co-opted_retiral}.  
    \end{legal}
%53
\item The minimum number of charity trustees is 6.
\end{legal}

\addsubsec{Eligibility}
\begin{legal}
%54
\item A person shall not be eligible for election/appointment to the board under clauses \ref{clause:agm_elect_trustee} to \ref{clause:trustee_re-election} unless he/she is a member of the organisation; a person appointed to the board under clause \ref{clause:board_co-opt_trustee} need not, however, be a member of the organisation.
%55
\item \label{clause:ineligible_trustee} A person will not be eligible for election or appointment to the board if he/she is:
    \begin{legal}
    \item disqualified from being a charity trustee under the Charities and Trustee Investment (Scotland) Act 2005; or
    \item an employee of the organisation.
    \end{legal}
\end{legal}

\addsubsec{Initial charity trustees}
\begin{legal}
%56
\item \label{clause:initial_trustees} The individuals who signed the charity trustee declaration forms which accompanied the application for incorporation of the organisation shall be deemed to have been appointed by the members as charity trustees with effect from the date of incorporation of the organisation.
\end{legal}

\addsubsec{Election, retiral, re-election}
\begin{legal}
%57
\item \label{clause:agm_elect_trustee} At each AGM, the members may elect any member (unless he/she is debarred from membership under clause \ref{clause:ineligible_trustee}) to be a charity trustee.
%58
\item \label{clause:board_appoint_trustee} The board may at any time appoint any member (unless he/she is debarred from membership under clause \ref{clause:ineligible_trustee}) to be a charity trustee.
%59
\item At each AGM, all of the charity trustees elected/appointed under clauses \ref{clause:agm_elect_trustee} and \ref{clause:board_appoint_trustee} (and, in the case of the first AGM, those deemed to have been appointed under clause \ref{clause:initial_trustees}) shall retire from office – but shall then be eligible for re-election under clause \ref{clause:agm_elect_trustee}. 
%60
\item \label{clause:trustee_re-election} A charity trustee retiring at an AGM will be deemed to have been re-elected unless:
    \begin{legal}
    \item he\slash she advises the board prior to the conclusion of the AGM that he\slash she does not wish to be re-appointed as a charity trustee; or
    \item an election process was held at the AGM and he/she was not among those elected/re-elected through that process; or 
    \item a resolution for the re-election of that charity trustee was put to the AGM and was not carried.
    \end{legal}
\end{legal}

\addsubsec{Appointment/re-appointment of co-opted charity trustees}
\begin{legal}
%61
\item \label{clause:board_co-opt_trustee} In addition to their powers under clause \ref{clause:board_appoint_trustee}, the board may at any time appoint any non-member of the organisation to be a charity trustee (subject to clause \ref{clause:max_trustees}, and providing he/she is not debarred from membership under clause \ref{clause:ineligible_trustee}) either on the basis that he/she has been nominated by a body with which the organisation has close contact in the course of its activities or on the basis that he/she has specialist experience and/or skills which could be of assistance to the board.
%62
\item \label{clause:co-opted_retiral} At each AGM, all of the charity trustees appointed under clause \ref{clause:board_co-opt_trustee} shall retire from office – but shall then be eligible for re-appointment under that clause.
\end{legal}

\addsubsec{Termination of office}

\begin{legal}
%63
\item \label{clause:cease_automatically} A charity trustee will automatically cease to hold office if:
    \begin{legal}
        \item he/she becomes disqualified from being a charity trustee under the Charities and Trustee Investment (Scotland) Act 2005;
        \item he/she becomes incapable for medical reasons of carrying out his/her duties as a charity trustee - but only if that has continued (or is expected to continue) for a period of more than six months;
        \item (in the case of a charity trustee elected/appointed under clauses \ref{clause:agm_elect_trustee} to \ref{clause:trustee_re-election}) he/she ceases to be a member of the organisation;
        \item he/she becomes an employee of the organisation;
        \item he/she gives the organisation a notice of resignation, signed by him/her;
        \item \label{clause:awol} he/she is absent (without good reason, in the opinion of the board) from more than three consecutive meetings of the board - but only if the board resolves to remove him/her from office;
        \item \label{clause:breach_of_conduct} he/she is removed from office by resolution of the board on the grounds that he/she is considered to have committed a material breach of the code of conduct for charity trustees (as referred to in clause \ref{clause:trustee_code_of_conduct});
        \item \label{clause:breach_of_duties} he/she is removed from office by resolution of the board on the grounds that he/she is considered to have been in serious or persistent  breach of his/her duties under section 66(1) or (2) of the Charities and Trustee Investment (Scotland) Act 2005; or
        \item \label{clause:removed_by_resolution} he/she is removed from office by a resolution of the members passed at a members’ meeting.
    \end{legal}
%64
\item A resolution under clause \ref{clause:cease_automatically}.\ref{clause:awol}, \ref{clause:cease_automatically}.\ref{clause:breach_of_conduct}, \ref{clause:cease_automatically}.\ref{clause:breach_of_duties} or \ref{clause:cease_automatically}.\ref{clause:removed_by_resolution} shall be valid only if:
    \begin{legal}
        \item the charity trustee who is the subject of the resolution is given reasonable prior written notice of the grounds upon which the resolution for his/her removal is to be proposed;
        \item the charity trustee concerned is given the opportunity to address the meeting at which the resolution is proposed, prior to the resolution being put to the vote; and
        \item (in the case of a resolution under clause \ref{clause:cease_automatically}.\ref{clause:removed_by_resolution}) at least two thirds (to the nearest round number) of the charity trustees then in office vote in favour of the resolution.
    \end{legal}
\end{legal}

\addsubsec{Register of charity trustees}
%65
\begin{legal}
\item The board must keep a register of charity trustees, setting out
    \begin{legal}
    \item for each current charity trustee:
        \begin{legal}
        \item his/her full name and address; 
        \item the date on which he/she was appointed as a charity trustee; and
        \item any office held by him/her in the organisation;
        \end{legal}
    \item for each former charity trustee - for at least 6 years from the date on which he/she ceased to be a charity trustee:
        \begin{legal}
        \item the name of the charity trustee;
        \item any office held by him/her in the organisation; and
        \item the date on which he/she ceased to be a charity trustee.
        \end{legal}
    \end{legal}
%66
\item The board must ensure that the register of charity trustees is updated within 28 days of any change:
    \begin{legal}
    \item which arises from a resolution of the board or a resolution passed by the members of the organisation; or 
    \item which is notified to the organisation.
    \end{legal}

%67
\item If any person requests a copy of the register of charity trustees, the board must ensure that a copy is supplied to him/her within 28 days, providing the request is reasonable; if the request is made by a person who is not a charity trustee of the organisation, the board may provide a copy which has the addresses blanked out - if the SCIO is satisfied that including that information is likely to jeopardise the safety or security of any person or premises.
\end{legal}

\addsubsec{Office-bearers}
\begin{legal}
%68
\item \label{clause:office-bearers} The charity trustees must elect (from among themselves) a chair, a vice-chair, a treasurer and a secretary.

%69
\item \label{clause:extra_office-bearers} In addition to the office-bearers required under clause \ref{clause:office-bearers}, the charity trustees may elect (from among themselves) further office-bearers if they consider that appropriate.

%70
\item All of the office-bearers will cease to hold office at the conclusion of each AGM, but may then be re-elected under clause \ref{clause:office-bearers} or \ref{clause:extra_office-bearers}.

%71
\item A person elected to any office will automatically cease to hold that office:
    \begin{legal}
        \item if he/she ceases to be a charity trustee; or 
        \item if he/she gives to the organisation a notice of resignation from that office, signed by him/her.
    \end{legal}
\end{legal}

\addsubsec{Powers of board}

\begin{legal}
%72
\item Except where this constitution states otherwise, the organisation (and its assets and operations) will be managed by the board; and the board may exercise all the powers of the organisation.
%73
\item A meeting of the board at which a quorum is present may exercise all powers exercisable by the board.
%74
\item The members may, by way of a resolution passed in compliance with clause \ref{clause:two_thirds_maj} (requirement for two-thirds majority), direct the board to take any particular step or direct the board not to take any particular step; and the board shall give effect to any such direction accordingly.
\end{legal}

\addsubsec{Charity trustees - general duties}
\begin{legal}
%75
\item \label{clause:trustees_duty} Each of the charity trustees has a duty, in exercising functions as a charity trustee, to act in the interests of the organisation; and, in particular, must:
    \begin{legal}
        \item seek, in good faith, to ensure that the organisation acts in a manner which is in accordance with its purposes;
        \item act with the care and diligence which it is reasonable to expect of a person who is managing the affairs of another person;
        \item in circumstances giving rise to the possibility of a conflict of interest between the organisation and any other party:
        \begin{legal}
            \item put the interests of the organisation before that of the other party;
            \item where any other duty prevents him/her from doing so, disclose the conflicting interest to the organisation and refrain from participating in any deliberation or decision of the other charity trustees with regard to the matter in question;
        \end{legal}
        \item ensure that the organisation complies with any direction, requirement, notice or duty imposed  under or by virtue of the Charities and Trustee Investment (Scotland) Act 2005.
    \end{legal}

%76
\item In addition to the duties outlined in clause \ref{clause:trustees_duty}, all of the charity trustees must take such steps as are reasonably practicable for the purpose of ensuring:
    \begin{legal}
        \item that any breach of any of those duties by a charity trustee is corrected by the charity trustee concerned and not repeated; and
        \item that any trustee who has been in serious and persistent breach of those duties is removed as a trustee.
    \end{legal}
%77
\item Provided he/she has declared his/her interest - and has not voted on the question of whether or not the organisation should enter into the arrangement - a charity trustee will not be debarred from entering into an arrangement with the organisation in which he/she has a personal interest; and (subject to clause \ref{clause:not_employee} and to the provisions relating to remuneration for services contained in the Charities and Trustee Investment (Scotland) Act 2005), he/she may retain any personal benefit which arises from that arrangement.

%78
\item \label{clause:not_employee} No charity trustee may serve as an employee (full time or part time) of the organisation; and no charity trustee may be given any remuneration by the organisation for carrying out his/her duties as a charity trustee. 

%79
\item The charity trustees may be paid all travelling and other expenses reasonably incurred by them in connection with carrying out their duties; this may include expenses relating to their attendance at meetings.
\end{legal}

\addsubsec{Code of conduct for charity trustees}
\begin{legal}
%80
\item \label{clause:trustee_code_of_conduct} Each of the charity trustees shall comply with the code of conduct (incorporating detailed rules on conflict of interest) prescribed by the board from time to time.

%81
\item The code of conduct referred to in clause \ref{clause:trustee_code_of_conduct} shall be supplemental to the provisions relating to the conduct of charity trustees contained in this constitution and the duties imposed on charity trustees under the Charities and Trustee Investment (Scotland) Act 2005; and all relevant provisions of this constitution shall be interpreted and applied in accordance with the provisions of the code of conduct in force from time to time
\end{legal}

\addsec[DECISION-MAKING BY THE CHARITY TRUSTEES]
{DECISION-MAKING BY THE\\ CHARITY TRUSTEES}
\addsubsec{Notice of board meetings}
\begin{legal}
%82
\item Any charity trustee may call a meeting of the board or ask the secretary to call a meeting of the board.
%83
\item At least 7 days' notice must be given of each board meeting, unless (in the opinion of the person calling the meeting) there is a degree of urgency which makes that inappropriate.
%84
\item If charity trustees are to be permitted to participate in a board meeting by way of audio and/or audio-visual link(s), the charity trustees must, in advance of the meeting, be provided with details of how to connect and participate via that link or links; and (particularly for the benefit of those charity trustees who may have difficulties in using a computer or laptop for this purpose) the charity trustees' attention should be drawn to the following options: 
    \begin{legal}[label=\alph*)]
    \item participating in the meeting via an audio link accessed by phone, using dial-in details (if that forms part of the arrangements); 
    \item (where attendance in person is to be permitted, either on an open basis or subject to a restriction on the total number who will be permitted to attend) the ability to attend the meeting in person.
    \end{legal}
\end{legal}

\addsubsec{Procedure at board meetings}
\begin{legal}
%85
\item \label{clause:board_quorum} No valid decisions can be taken at a board meeting unless a quorum is present; the quorum for board meetings is 6 charity trustees, present in person or by proxy.
%86
\item An individual participating in a board meeting via an audio or audio visual link will be deemed to be present in person (or, if they are not a charity trustee, will be deemed to be in attendance) at the meeting.
%87
\item If at any time the number of charity trustees in office falls below the number stated as the quorum in clause \ref{clause:board_quorum}, the remaining charity trustee(s) will have power to fill the vacancies or call a members' meeting - but will not be able to take any other valid decisions.

%88
\item The chair of the organisation, who must be an elected trustee, should act as chairperson of each board meeting.
%89
\item If the chair is not present within 15 minutes after the time at which the meeting was due to start (or is not willing to act as chairperson), the charity trustees present at the meeting must elect (from among the elected trustees present) the person who will act as chairperson of that meeting.
%90
\item Every charity trustee has one vote, which can be given personally or by proxy. 
%91
\item All decisions at board meetings will be made by majority vote.
%92
\item If there is an equal number of votes for and against any resolution, the chairperson of the meeting, who must be an elected trustee, will be entitled to a second (casting) vote.
%93
\item The board may if they consider appropriate allow charity trustees to participate in board meetings by way of an audio and/or audio-visual link or links, providing: 
    \begin{legal}[label=\alph*)]
    \item the means by which charity trustees can participate via that link or links are not subject to technical complexities, significant costs or other factors which are likely to represent for all, or a significant proportion, of the charity trustees - a barrier to participation; and 
    \item the manner in which the meeting is conducted ensures, so far as reasonably possible, that those charity trustees who participate via an audio or audio-visual link are not disadvantaged with regard to their ability to contribute to discussions at the meeting, as compared with those charity trustees (if any) who are attending in person (and vice versa).
    \end{legal}
%94
\item If restrictions arising from public health legislation or guidance are likely to mean that attendance in person at a proposed board meeting would not be possible or advisable for one or more of the charity trustees, the board must make arrangements for charity trustees to participate in that board meeting by way of audio and/or audio-visual link(s) which allow them to hear and contribute to discussions at the meeting; and on the basis that: 
    \begin{legal}[label=\alph*)]
    \item the requirements set out above will apply; and 
    \item the board must use all reasonable endeavours to ensure that all charity trustees have access to one or more means by which they may hear and contribute to discussions at the meeting.
    \end{legal}
%95 
\item A board meeting may involve two or more charity trustees participating via attendance in person while other charity trustees participate via audio and/or audio-visual links; or it may involve participation solely via audio and/or audio-visual links.
%96
\item Where a charity trustee is participating in a board meeting via an audio or audio-visual link, they may cast their vote on a given resolution orally, or by way of some form of visual indication, or by use of a voting button or similar, or by way of a message sent electronically. 
%97
\item The board may, at its discretion, allow any person to attend (whether in person or by way of an audio or audio-visual link) and speak at a board meeting notwithstanding that they are not a charity trustee - but on the basis that they must not participate in decision-making.
%98 [duplicate removed!]
%99
\item \label{clause:vote_conflict} A charity trustee must not vote at a board meeting (or at a meeting of a sub-committee) on any resolution which relates to a matter in which he/she has a personal interest or duty which conflicts (or may conflict) with the interests of the organisation; he/she must withdraw from the meeting while an item of that nature is being dealt with.

\item For the purposes of clause \ref{clause:vote_conflict}:
    \begin{legal}
        \item an interest held by an individual who is “connected” with the charity trustee under section 68(2) of the Charities and Trustee Investment (Scotland) Act 2005 (husband/wife, partner, child, parent, brother/sister etc) shall be deemed to be held by that charity trustee;
        \item a charity trustee will be deemed to have a personal interest in relation to a particular matter if a body in relation to which he/she is an employee, director, member of the management committee, officer or elected representative has an interest in that matter.
    \end{legal}
\end{legal}

\addsubsec{Technical objections to remote participation in board meetings}
\begin{legal}
%101
\item This constitution imposes certain requirements regarding the use of audio and/or audio-visual links as a means of participation and voting at board meetings; providing the arrangements made by the board in relation to a given board meeting (and the manner in which the meeting is conducted) are consistent with those requirements: 
    \begin{legal}[label=\alph*)]
    \item a charity trustee cannot insist on participating in the board meeting, or voting at the board meeting, by any particular means; 
    \item the board meeting need not be held in any particular place; 
    \item the board meeting may be held without any particular number of those participating in the meeting being present in person at the same place (but, notwithstanding that, the quorum requirements - taking account of those participating via audio and/or audio-visual links - must still be met); 
    \item the board meeting may be held by any means which permits those participating in the meeting to hear and contribute to discussions at the meeting; 
    \item a charity trustee will be able to exercise the right to vote at the board meeting by such means as is determined by the chairperson of the meeting (consistent with the arrangements vote to be taken into account in determining whether or not a resolution is passed.
    \end{legal}
\end{legal}

\addsubsec{Resolutions agreed by the board in writing or by e-mail}
\begin{legal}
%102
\item \label{clause:email_resolution} A resolution agreed to in writing (or by e-mail) by a majority of the charity trustees then in office shall (subject to the above) be as valid as if duly passed at a board meeting. A resolution under clause \ref{clause:email_resolution} shall not be valid unless a copy of the resolution was circulated to all of the charity trustees, along with a cut-off time (which must be reasonable in the circumstances) for notifications under clause \ref{clause:email_resolution}. If a resolution is circulated to the charity trustees under clause \ref{clause:email_resolution}, any one or more charity trustees may, following receipt of a copy of the resolution, notify the secretary that they consider that a board meeting should be held to discuss the matter which is the subject of the resolution; and if any such notification is received by the secretary prior to the cut-off time: 
    \begin{legal}[label=\alph*)]
    \item the secretary must convene a board meeting accordingly, and on the basis that it will take place as soon as reasonably possible;
    \item the resolution cannot be treated as valid under clause \ref{clause:email_resolution} unless and until that board meeting has taken place; 
    \item the board may (if they consider appropriate, on the basis of the discussions at the meeting) resolve at that board meeting that the resolution should be treated as invalid, notwithstanding that it had previously been agreed to in writing (or by e-mail) by a majority of the charity trustees then in office.
    \end{legal}
\end{legal}

\addsubsec{Minutes}
\begin{legal}
%103
\item \label{clause:keep_minutes_board} The board must ensure that proper minutes are kept in relation to all board meetings and meetings of sub-committees.
%104
\item The minutes to be kept under clause \ref{clause:keep_minutes_board} must include the names of those present; and (so far as possible) should be signed by the chairperson of the meeting.
%105
\item \label{clause:make_minutes_available} The board shall (subject to clause \ref{clause:minutes_confidential}) make available copies of the minutes referred to in clause \ref{clause:keep_minutes_board} to any member of the public requesting them.
%106
\item \label{clause:minutes_confidential} The board may exclude from any copy minutes made available to a member of the public under clause \ref{clause:make_minutes_available} any material which the board considers ought properly to be kept confidential - on the grounds that allowing access to such material could cause significant prejudice to the interests of the organisation or on the basis that the material contains reference to employee or other matters which it would be inappropriate to divulge.
\end{legal}

\addsec{ADMINISTRATION}
\addsubsec{Delegation to sub-committees}
\begin{legal}
%107
\item \label{clause:delegate_subcommittee} The board may delegate any of their powers to sub-committees; a sub-committee must include at least one charity trustee, but other members of a sub-committee need not be charity trustees.
%108
\item \label{clause:delegate_chair} The board may also delegate to the chair of the organisation (or the holder of any other post) such of their powers as they may consider appropriate.
%109
\item When delegating powers under clause \ref{clause:delegate_subcommittee} or \ref{clause:delegate_chair}, the board must set out appropriate conditions (which must include an obligation to report regularly to the board).

%110
\item Any delegation of powers under clause \ref{clause:delegate_subcommittee} or \ref{clause:delegate_chair} may be revoked or altered by the board at any time.
%111
\item The rules of procedure for each sub-committee, and the provisions relating to membership of each sub-committee, shall be set by the board.
\end{legal}

\addsubsec{Operation of accounts}
\begin{legal}
\item \label{clause:two_signatories} Subject to clause \ref{clause:electronic_signatories}, the signatures of two out of three signatories appointed by the board will be required in relation to all operations (other than the lodging of funds) on the bank and building society accounts held by the organisation; at least one out of the two signatures must be the signature of a charity trustee.

\item \label{clause:electronic_signatories} Where the organisation uses electronic facilities for the operation of any bank or building society account, the authorisations required for operations on that account must be consistent with the approach reflected in clause \ref{clause:two_signatories}.
\end{legal}

\addsubsec{Accounting records and annual accounts}
\begin{legal}
\item The board must ensure that proper accounting records are kept, in accordance with all applicable statutory requirements.
\item The board must prepare annual accounts, complying with all relevant statutory requirements; if an audit is required under any statutory provisions (or if the board consider that an audit would be appropriate for some other reason), the board should ensure that an audit of the accounts is carried out by a qualified auditor.
\end{legal}

\addsec{MISCELLANEOUS}
\addsubsec{Winding-up}
\begin{legal}
\item The organisation will run on a not for profit basis. If the organisation is to be wound up or dissolved, the winding-up or dissolution process will be carried out in accordance with the procedures set out under the Charities and Trustee Investment (Scotland) Act 2005. 
\item Any surplus assets available to the organisation immediately preceding its winding up or dissolution must be used for purposes which are the same as - or which closely resemble - the purposes of the organisation as set out in this constitution.
\end{legal}

\addsubsec{Alterations to the constitution}
%118
\begin{legal}
\item This constitution may (subject to clause \ref{clause:changes_req_oscr}) be altered by resolution of the members passed at a members’ meeting (subject to achieving the two thirds majority referred to in clause \ref{clause:two_thirds_maj}) or by way of a written resolution of the members.  
%119
\item \label{clause:changes_req_oscr} The Charities and Trustee Investment (Scotland) Act 2005 prohibits taking certain steps (eg change of name, an alteration to the purposes, amalgamation, winding-up) without the consent of the Office of the Scottish Charity Regulator (OSCR).
\end{legal}

\addsubsec{Interpretation}
\begin{legal}
%120
\item \label{clause:charities_act} References in this constitution to the Charities and Trustee Investment (Scotland) Act 2005 should be taken to include:
    \begin{legal}
    \item \label{clause:charities_act_mods} any statutory provision which adds to, modifies or replaces that Act; and 
    \item any statutory instrument issued in pursuance of that Act or in pursuance of any statutory provision falling under clause \ref{clause:charities_act}.\ref{clause:charities_act_mods} above.
    \end{legal}
%121
\item In this constitution:
    \begin{legal}
    \item “charity” means a body which is either a “Scottish charity” within the meaning of section 13 of the Charities and Trustee Investment (Scotland) Act 2005 or a “charity” within the meaning of section 1 of the Charities Act 2011, providing (in either case) that its objects are limited to charitable purposes;
    \item “charitable purpose” means a charitable purpose under section 7 of the Charities and Trustee Investment (Scotland) Act 2005 which is also regarded as a charitable purpose in relation to the application of the Taxes Acts.
    \end{legal}
\end{legal}

\end{document}





