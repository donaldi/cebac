\documentclass[a4paper,11pt,onecolumn ]{article}

% NB this file is under git version control at 
% https://github.com/donaldi/cebac
% so if you have not downloaded it from there you may not have
% the latest version!

\usepackage{multicol}
\newcommand{\namesigdate}[2][5cm]{%
  \begin{tabular}{@{}p{#1}@{}}
    #2 \\[2\normalbaselineskip] \hrule \\[0pt]
    {\small \textit{Name}} \\[2\normalbaselineskip] \hrule \\[0pt]
    {\small \textit{Signature}} \\[2\normalbaselineskip] \hrule \\[0pt]
    {\small \textit{Date}}
  \end{tabular}
}

%extend lists
\usepackage{enumitem}

% lets sections/subsections be un-numbered and yet still in toc
% note that I've slightly modified this to work for subsecs too
\usepackage[silent]{unnumberedtotoc}

% use images
\usepackage{graphicx}
\graphicspath{ {./} }

%hyperlink refs
\usepackage{hyperref}
\usepackage{xcolor}
\definecolor{ultramarine}{RGB}{0,32,96} 
\hypersetup{
    colorlinks = true,
    urlcolor = ultramarine,
    linkcolor = ultramarine,
    citecolor = red
}

%\usepackage{draftwatermark}
%\SetWatermarkText{Draft}
%\SetWatermarkScale{5}

\title{%
  Employee Contract \\
  COMANN EACHDRAIDH SGÌRE A' BHAC \\
  \large Statement of Terms and Conditions of Employment \\
  JANE DAVIDSON
  }

\pagestyle{plain}

\newlist{legal}{enumerate}{10}

% set "legal" list (cloned from enumerate) to use arabic bullets, have no left-margin and 
% in references (ref)to use the number without the period after it.
\setlist[legal]{label*=\arabic*.,ref=\arabic*,left=0pt}
% make the main global list resumable
\setlist[legal,1]{resume,widest=100}


\begin{document}
\maketitle

\begin{center}
\includegraphics[width=0.3\paperwidth]{cebac-logo}
\end{center}

\pagebreak

\tableofcontents

\pagebreak

\addsec{TERMS AND CONDITIONS OF EMPLOYMENT}

\textbf{BETWEEN}
\begin{enumerate}[label=\alph*)]
\item \textbf{COMANN EACHDRAIDH SGÌRE A' BHAC SCIO}, an association whose registered office is at 
Heritage Centre, School Road, Vatisker Isle of Lewis HS2 0JY (hereinafter referred to as 
``the Employer''), and

\item \textbf{JANE DAVIDSON} (hereinafter referred to as ``you'').
\end{enumerate}

\textbf{IT IS AGREED} as follows:



\begin{legal}
\item 
\addsubsec{General}
This document is a statement of the main terms and conditions of employment
which govern your service with the Employer. Your service with the Employer is
also subject to the terms contained in the letter offering you employment 'the offer
letter'. If there should be any ambiguity or discrepancy between the terms in the
offer letter and in the terms set out in this document, the terms of the offer letter
will prevail, except where expressly stated to the contrary.
\end{legal}



\begin{legal}
  \item \addsubsec{Duties and Job Title}
  \begin{legal}
  \item You are employed by the Employer in the capacity of \textbf{development officer}. You will
  be required to undertake as outlined in the job description and duties and
  responsibilities as may be determined by the Employer from time to time.
  \item The Employer reserves the right to vary your duties and responsibilities at any
  time and from time to time according to the needs of the Employer’s business,
  following discussion and agreement with you.
  \end{legal}
\end{legal}



\begin{legal}
\item \addsubsec{Date of commencement, date of continuous employment and notice period}
\begin{legal}
  \item Your period of continuous employment with us begins on \textbf{3 December 2024}.
  \item No employment with a previous employer counts as part of your period of
  continuous employment.
  \item Your employment is for a fixed term and will terminate on \textbf{2 December 2025}.
  Subject to funding, this contract may be extended. Your employment may be
  terminated at any time before its expiry by either party giving to the other 30
  days’ notice in writing of the termination of your employment. Alternatively,
  your employment may be summarily terminated where you are found guilty of
  gross misconduct.
  \item In accepting your appointment it shall be deemed that you have accepted all the
  terms and conditions set out in this Contract.
  \item This Contract of Employment annuls any previous agreement whether verbal or
  written given to you at any time.
\end{legal}


\item \addsubsec{Hours of work}
\begin{legal}
\item Your normal working hours are between 09:00 am and 5:00 pm Tuesdays to
Thursdays inclusive, with an hour for lunch.
\item The Employer reserves the right to alter working hours as necessary, following
discussion and agreement with you.
\item You may be asked to work additional hours beyond your normal hours and it is
a condition of your employment that you agree to do so when reasonably asked.
You will not be entitled to overtime payments for hours worked outside your
normal working hours.
\end{legal}


\item \addsubsec{Place of work}
You will be expected to work from the Employer's office (the Heritage Centre) on 
Thursdays. You may work remotely on the other days.

\item \addsubsec{Remuneration}
\begin{legal}
\item Your salary is £15,000 per year (£25,000 pro rata), to be paid monthly normally on 
last weekday of every month. Payment will be made direct credit transfer to a bank or
building society account nominated by you. You will not be entitled to overtime
payment for hours worked outside your normal weekly hours (as specified
above).
\item Your salary will be reviewed annually entirely at our discretion.
\item The Employer reserves the right to seek reimbursement by deduction from your
salary, in accordance with the provisions of the Employment Rights Act 1966 in
the event of any material deficiencies attributable to you, in particular damage
to Employer property or in the event of overpayment of salary, recovery of
unearned holiday pay or other remunerations, or if any other sums are due by
you to the Employer arising from your employment.
\end{legal}


\item \addsubsec{Collective agreements}
There are no collective agreements relevant to your employment.

\item \addsubsec{Holidays}
\begin{legal}
\item You are entitled to 17 days holiday in each complete calendar year, 
including bank and public holidays. Special bank holidays, which may be given at the
Employer's discretion, are not included in this allowance. 
\item The holiday year commences on \textbf{3 December} each year.
\item If your employment commences or finishes part way through the holiday year,
your holiday entitlement will be prorated accordingly.

\item If, on termination of employment:

\begin{legal}
  \item you have exceeded your prorated holiday entitlement; the Employer will
  deduct a payment in lieu of days holiday taken in excess of your prorated
  holiday entitlement, and you authorise the Employer to make a deduction
  from the payment of any final salary.
  \item you have holiday entitlement still owing, the Employer may, at its
  discretion, require you to take your holiday during your notice period or
  make a payment in lieu of untaken holiday entitlement.
\end{legal}

\item Holidays must be taken at times convenient to the Employer. You must obtain
approval of proposed holiday dates in advance from the Chairperson or Vice
Chair. You will not be allowed to take more than two weeks at any one time,
save at the Employer’s discretion. You must not book holidays until your
request for approval has been formally agreed.
\item All holidays must be taken in the year in which it is accrued. In exceptional
circumstances you may carry forward up to 5 days untaken holiday entitlement
to the next holiday year. This applies for one year only, and holidays may not be
carried forward to a subsequent holiday year.
\item If you are sick or injured while on holiday, the Employer will allow you to
transfer to sick leave and take replacement holiday at a later date. This is
strictly subject to the following:
\begin{legal}
  \item You must contact the Chairperson or Vice Chair in accordance with the
  notification of sickness absence procedure as soon as you know that your
  holiday will be affected by sickness or injury;
  \item The full period of your incapacity due to sickness or injury must be
  certificated by a qualified medical practitioner, and
  \item Within 5 days of your return to work, you must confirm in writing how
  much of your holiday was affected by sickness or injury and the amount of
  leave you wish to take at another time. This written notification must be
  sent to the Chairperson or Vice Chair.
\end{legal}
\end{legal}

\item \addsubsec{Sickness Absence}

\begin{legal}
\item In the event of your absence for any reason you or someone on your behalf
should contact the Chairperson or Vice Chair at the earliest opportunity and no
later than 11:00 am on the first day of the absence to inform them of the reason
for absence. You must inform the Employer as soon as possible of any change in
the date of your expected return to work.
\item A self-certification form should be completed for absences of up to seven days.
The form will be supplied to you.
\item For periods of sickness of more than seven consecutive days, including
weekends, you will be required to obtain a Statement of Fitness for Work (‘Fit
Note’) / Medical Certificate and send this to the Chairperson or Vice Chair. A Fit
Note / Medical Certificate should be sent to the Employer to cover the period of
your sickness absence from work.
\item If you are absent for four or more days by reason of sickness or incapacity, you
are entitled to Statutory Sick Pay (SSP), provided that you have met the
requirements above. For the purposes of the SSP scheme the ‘qualifying days’
are Monday to Friday. There is no contractual right to payment in respect of
periods of absence due to sickness or incapacity. Any such payments are at the
discretion of the Employer.
\item The Employer has the right to monitor and record absence levels and reasons
for absences. Such information will be kept confidential.
\item The Employer may require you to undergo a medical examination by a medical
practitioner nominated by us at any stage of your employment, and you agree to
authorise such medical practitioner to prepare a report detailing the results of
the examination, which you agree may be disclosed to the Employer. The
Employer will bear the cost of such medical examination. Such an examination
will only be requested by the Employer where it is reasonable to do so.
\end{legal}

\item \addsubsec{Maternity and Paternity Rights}
The Employer will comply with its statutory obligations with respect to maternity
and paternity rights and rights dealing with time off for dependants. The Employer’s
policies in this regard are available on request from the Chairperson.

\item \addsubsec{Pension}
\begin{legal}
  \item If you are eligible, the Employer will auto-enrol you into a pension scheme, in
accordance with the Employer’s pension auto-enrolment obligations.
Full details of the scheme will be provided when you are enrolled, including the
minimum contribution level that you will be required to make and your right to
opt out if you do not want to join the scheme. While participating in the scheme,
you agree to worker pension contributions being deducted from your salary.
The scheme is subject to its rules as may be amended from time to time, and the
Employer may replace the scheme with another pension scheme at any time.
\end{legal}

\item \addsubsec{Non–Compulsory Retirement}
The Employer does not operate a normal retirement age and therefore you will not
be compulsorily retired on reaching a particular age. However, you can choose to
retire voluntarily at any time, provided that you give the required period of notice to
terminate your employment.

\item \addsubsec{Restrictions and Confidentiality}
\begin{legal}
\item You may not, without the prior written consent of the Employer, devote any
time to any business other than the business of the Employer or to any
public or charitable duty or endeavour during your normal hours of work.
\item You will not at any time either during your employment or afterwards use
or divulge to any person, firm or Employer, except in the proper course of
your duties during your employment by the Employer, any confidential
information identifying or relating to the Employer, details of which are not
in the public domain.
\end{legal}

\item \addsubsec{Mobility}
You may be required to travel on Employer business anywhere in the UK. Travel
and subsistence will be paid to you in accordance with the Employer’s Expenses
Policy.

\item \addsubsec{Grievance Procedure}
The formal Grievance Procedure is available on request from the Chairperson.

\item \addsubsec{Disciplinary Procedure}
The disciplinary rules applicable to your employment are set out in the attached
Disciplinary Rules and Procedure.

\item \addsubsec{Employee Handbook and Employment Policies}
All employees have a duty to adhere to the Employer’s other policies in force,
including but not exclusive to the Employer’s Health and Safety, Fire Safety, Sickness
and Absence and Equal Opportunities Policies.

\item \addsubsec{Termination of employment}
\begin{legal}
  \item During the 1 month probationary period the notice required by either party to
  this Contract to terminate your employment will be one week.
  After the successful completion of any probationary period, your employment may
  be ended by you giving the Employer one month’s written notice. The Employer will
  give you one month’s written notice and after four years’ continuous service a further
  one week’s notice for each additional complete year of service up to a maximum of 12
  weeks’ notice.
  \item The Employer reserves the right in their absolute discretion to pay you salary
  in lieu of notice.
  \item Nothing in this Contract prevents the Employer from terminating your
  employment summarily or otherwise in the event of any serious breach by you
  of the terms of your employment or in the event of any act or acts of gross
  misconduct by you.
\end{legal}

\item \addsubsec{Data Protection}
You agree to the Employer holding and processing, both electronically and
manually, personal data about you (including sensitive personal data as defined in
the Data Protection Act 1998) for the operations, management, security or
administration of the Employer and for the purpose of complying with applicable
laws, regulations and procedures.

\item \addsubsec{Confidential Information}
You will not at any time either during your employment or afterwards use or
divulge to any person, firm or Employer, except in the proper course of your duties
during your employment by the Employer, any confidential information identifying
or relating to the Employer, details of which are not in the public domain.

\item \addsubsec{Copyright, Inventions and Patents}
\begin{legal}
\item All records, documents, papers (including copies and summaries thereof) and
other copyright protected works made or acquired by you in the course of
your employment shall, together with all the world-wide copyright and design
rights in all such works, be and at all times remain the absolute property of
the Employer.
\item You hereby irrevocably and unconditionally waive all rights granted by
Chapter IV of Part I of the Copyright, Designs and Patents Act 1988 that vest in
you (whether before, on or after the date hereof) in connection with your
authorship of any copyright works in the course of your employment with the
Employer, wherever in the world enforceable, including without limitation the
right to be identified as the author of any such works and the right not to have
any such works subjected to derogatory treatment.
\end{legal}

\item \addsubsec{Changes to Terms and Conditions of Employment}
The Employer may amend, vary or terminate the terms and conditions in this
document. Any such change to your
terms and conditions will be subject to consultation and agreement with you and
notified to you personally in writing.

\item \addsubsec{Severability}
The various provision of this Agreement are severable, and if any provision or
identifiable part thereof is held to be invalid or unenforceable by any court of
competent jurisdiction then such invalidity or unenforceability shall not affect the
validity or enforceability of the remaining provisions or identifiable parts.

\item \addsubsec{Jurisdiction}
This Agreement shall be governed by and construed in accordance with Scots Law
and Scottish Courts.
\end{legal}

\newpage


\begin{center}
    
\begin{multicols}{2}
\begin{center}
Issued for and on behalf of \textbf{COMANN EACHDRAIDH SGÌRE A' BHAC SCIO} by the
\textbf{chair} of the trustees.
\end{center}


\noindent
\namesigdate{}

\columnbreak

\begin{center}
\textbf{JANE DAVIDSON}
I hereby warrant and confirm that I am not prevented by previous employment terms
and conditions, or in any other way, from entering into employment with the Employer
or performing any of the duties of employment referred to above. I accept the terms of
this Agreement.
\end{center}

\noindent
\namesigdate{}


\end{multicols}

\end{center}





\end{document}





